
\chapter{Summary}\label{chap:summary}

In this thesis, we presented the results of theoretical and experimental investigation of the influence of chemical composition of semiconductor quantum dots on their electronic structure. Three QDs systems were studied, (i) InGaAs stress tuned QDs, (ii) InAs QDs capped by GaAsSb layer, and (iii) In$_{1-x}$Ga$_x$As$_y$Sb$_{1-y}$ QDs on GaP substrate with 5~ML GaAs interlayer. The theoretical study was performed using 8-band $\mathbf{k\cdot p}$ method and configuration interaction method provided by the supervisor. The experimental part was performed mainly by steady-state and time-resolved photoluminescence measurements. Information about size, shape and material contribution of QDs was collected during TEM and EDX experiments.

We pinpointed out the importance of non-linear piezoelectricity to electric dipole moment in InGaAs/GaAs quantum dots and showed that non-linear terms might dominate compared to linear ones. Moreover, we highlighted the necessity of a large built-in in-plane prestress created due to bonding the sample onto the piezo actuator. During $\mathbf{k\cdot p}$ calculations we found that the most important parameters influencing the electric dipole in QD are the height of dot and the bonding pre-stress. For these quantities we created analytical models originating from empirical observations which can be used for estimation of the electric dipole in quantum dots.

We have experimentally studied the excitonic structure of type-II InAs/GaAsSb/GaAs quantum dots by intensity and polarization resolved photoluminescence. In our spectra we identified multiparticle optical transitions of neutral exciton, biexciton, and negative trion. Different polarization enabled us to discriminate different recombination channels and revealed the coexistence of type-I and type-II confinement.

Moreover, blue-shift of type-II transitions in InAs/GaAsSb/GaAs has been analyzed. For the determined complexes, multi-exponential shift of emission energy with increasing excitation power was observed and modelled by self-consistent multi-particle calculations with added background potential in quantum dot area. This approximate method can successfully explain (i) blue-shift of uncapped QDs which is well explained by relation $E\propto P^{1/3}$, and (ii) blue-shift of capped QDs.

Finally, we found that $\Gamma$ and $L$ bands in In$_{1-x}$Ga$_x$As$_y$Sb$_{1-y}$/5~ML GaAs/GaP QDs for compositions around $x=0.9$ and $y=0.85$ swap in energies which was observed also in our photoluminescence measurements. These structures have significantly large blue-shift of the emission spectrum with pumping (more than 30~meV corresponding to change of excitation power by 4 magnitudes) contrary to the fact that they have type-I (in direct space) bandalignment. Optical transitions for these samples differ in radiative time component which for $\Gamma$-band was determined to be around 25~ns. The indirect (in reciprocal-space) transitions from $L$ band are slower and much more sensitive to material content. For $x=0.75$ and $y=0.95$ we have found decay times ~200~ns, while for $x=0.85$ and $y=0.85$ those were around 300~ns.


\newpage

\newpage 